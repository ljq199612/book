\documentclass{article}
\usepackage{tikz}
\usepackage{CJKutf8}
\usepackage{amsmath}
\usepackage{amsthm}
\begin{document}
\begin{CJK}{UTF8}{gbsn}
  \newtheorem*{Ex}{习题}
\begin{Ex}
      设$R$,$S$,$T$为任意三个集合,证明:$(R\bigtriangleup S)\cap (R\bigtriangleup T)\subseteq R\bigtriangleup (S\cap T)$。
    \end{Ex}

    \begin{proof}[证明(利用自然语言叙述)]
      对任意的$x\in (R\bigtriangleup S)\cap (R\bigtriangleup T)$,分两种情况讨论:

      1)如果$x\in R$,由$x\in R\bigtriangleup S=(S\setminus R)\cup (R\setminus S)$知$x \notin S$,从而$x\notin S\cap T$,此时$x\in R\bigtriangleup (S\cap T)$;

      2)如果$x\notin R$,由$x\in R\bigtriangleup S=(S\setminus R)\cup (R\setminus S)$知$x\in S$,由$x\in R\bigtriangleup T = (R\setminus T)\bigtriangleup (T\setminus R)$知$x\in T$,从而$x\in S\cap T$,$x\in R\bigtriangleup (S\cap T)$。

      综合以上两种情况,对任意的$x\in (R\bigtriangleup S)\cap (R\bigtriangleup T)$,$x\in R\bigtriangleup (S\cap T)$,结论得证。
      
    \end{proof}
    
\begin{proof}[证明(利用集合运算规则)]
  \begin{equation*}
    \begin{split}
  &(R\bigtriangleup S)\cap (R\bigtriangleup T)\\
  =&(R\setminus S \cup S\setminus R) \cap (R\setminus T \cup T\setminus R)\\
  =&(R\setminus S \cap R\setminus T)\cup (R\setminus S\cap T\setminus R)\cup (S\setminus R\cap R\setminus T)\cup (S\setminus R\cap T\setminus R)\\
  =&R\setminus (S\cup T)\cup \phi \cup \phi \cup (S\cap T)\setminus R\\
  =&R\setminus (S\cup T)\cup  (S\cap T)\setminus R\\
  \subseteq &R\setminus (S\cap T) \cup (S\cap T)\setminus R\\
  =&R\bigtriangleup (S\cap T)
\end{split}
\end{equation*}
\end{proof}
\begin{proof}[证明(利用符号逻辑)]
    \begin{equation*}
    \begin{split}
      \forall x, &x \in (R\bigtriangleup S)\cap (R\bigtriangleup T) \\
      \Leftrightarrow& x \in R\bigtriangleup S \land x\in R\bigtriangleup T\\
      \Leftrightarrow& x \in (R\setminus S)\cup (S\setminus R) \land x\in (R\setminus T)\cup (T\setminus R)\\
      \Leftrightarrow& ((x \in R \land x\notin S) \lor (x \in S \land x \notin R))\land ((x\in R \land x\notin T)\lor (x\in T \land x \notin R))\\
      \Leftrightarrow& (x \in R \land x \notin S  \land x\notin T) \lor (x \in S \land x \in T \land x\notin R)\\
      \Rightarrow & (x \in R \land (x\notin S \lor x\notin T)) \lor (x \in S \land x \in T \land x\notin R)\\
      \Leftrightarrow& (x \in R \land x \notin S\cap T)\lor ((x\in S\cap T)\land x\notin R)\\
      \Leftrightarrow& (x \in R  \setminus (S\cap T))\lor (x\in (S\cap T)\setminus R)\\
      \Leftrightarrow& x \in R  \bigtriangleup (S\cap T)\\      
    \end{split}
  \end{equation*}

\end{proof}

  \begin{Ex}
    以下结论是否成立,若成立,给出证明;若不成立,请说明理由。
    
      设$R$,$S$,$T$为任意三个集合,则$R\bigtriangleup (S\cap T) \subseteq (R\bigtriangleup S)\cap (R\bigtriangleup T)$。
\end{Ex}
\begin{proof}[解]
  该结论不成立。这是因为当$R=\{1\}, S=\{1\}, T = \phi$时,$R\bigtriangleup (S\cap T)=\{1\}$, $(R\bigtriangleup S)\cap (R\bigtriangleup T)=\phi$,$R\bigtriangleup (S\cap T)\subseteq (R\bigtriangleup S)\cap (R\bigtriangleup T)$不成立。
\end{proof}

\end{CJK}
\end{document}



%%% Local Variables:
%%% mode: latex
%%% TeX-master: t
%%% End:
