\documentclass{article}
\usepackage{tikz}
\usepackage{CJKutf8}
\usepackage{amsmath}
\usepackage{amsthm}

\newtheorem*{Example}{例}

\begin{document}
\begin{CJK}{UTF8}{gbsn}
  命题:可以判断真假的陈述句。通常,我们用$T$表示真,用$F$表示假。
  \begin{Example}\quad
    
    \begin{itemize}
    \item 2是偶数。(真命题)
    \item 北京是中国的首都。(真命题)
    \item 多伦多是加拿大的首都。(假命题)
    \item 这句话是假的。(不能判断真假的陈述句,不是命题)
    \end{itemize}
  \end{Example}
  谓词:命题的谓语部分。
  
  \begin{Example}\quad
    
    \begin{description}
    \item     [$P(x): x$ 是偶数] 这里$P$为一元谓词,表示“是偶数”。当$x$为某个确定的数字时,$P(x)$则对应一个命题。例如$P(2)$为真命题,$P(1)$为假命题。这里,$P$之所以被称为一元谓词,是因为$P(x)$只包含一个变量$x$。
    \item     [$P(x,y): x >y$]  这里$P$为二元谓词,表示$>$。当$x$和$y$为确定的数字时,$P(x,y)$则对应一个命题。例如$1>0$为真命题,$0>1$为假命题。这里,$P$之所以被称为二元谓词,是因为$P(x,y)$包含两个变量$x$和$y$。
    \end{description}
相应的,有三元谓词,四元谓词,......
\end{Example}

我们还可以用如下方式由谓词得到命题:

\begin{description}
\item [$\forall x P(x)$:] 对任意的$x$,$P(x)$。For All中的$A$上下颠倒可以得到$\forall$。
\item [$\exists x P(x)$:] 存在$x$,$P(x)$。There Exists中的$E$左右颠倒可以得到$\exists$。
\end{description}

命题可以由联结词$\lnot$,$\land$,$\lor$,$\to$,$\leftrightarrow$联结而构成复合命题。

设$p$为命题,则$\lnot p$表示“$p$不成立”。

 \begin{tabular}{c|c}
    p& $\lnot$ p\\
    \hline
    T&F\\
    F&T\\
  \end{tabular}

  设$p$和$q$为两个命题,则$p\land q$表示“$p$成立,并且$q$成立”。
  
  \begin{tabular}{cc|c}
    p& q& p $\land$ q\\
    \hline
    T&T&T\\
    T&F&F\\
    F&T&F\\
    F&F&F\\
  \end{tabular}

  设$p$和$q$为两个命题,则$p\lor q$表示“$p$成立,或者$q$成立”。
  
  \begin{tabular}{cc|c}
    p& q& p $\lor$ q\\
    \hline
    T&T&T\\
    T&F&T\\
    F&T&T\\
    F&F&F\\
  \end{tabular}

设$p$和$q$为两个命题,则$p\to q$表示“如果$p$成立,那么$q$成立”。  

    \begin{tabular}{cc|c}
    p& q& p $\to$ q\\
    \hline
    T&T&T\\
    T&F&F\\
    F&T&T\\
    F&F&T\\
    \end{tabular}\hspace{0.87cm}

    这里需要注意的是,当$p$为假时,则$p\to q$一定为真,这是所有数学家共同的约定。
    下面的例子可以帮助大家更好的理解其实我们已经用到了这个约定。

    对任意的实数$x$,当$x>1$时,$x^2 > 1$。该命题显然是真命题,可以符号化为$\forall x \; x > 1 \to x^2 > 1$。那么,既然对于任意的$x$,$x>1 \to x^2>1$成立,则

    1)当$x=2$时, $2 > 1 \to 2^2 >1$成立,这对应于以上真值表的第一行;

    2)当$x=0$时,$0 > 1 \to 0^2 > 1$成立,这对应于以上真值表的第四行;

    3)当$x=-2$时,$-2>1 \to (-2)^2 > 1$成立,这对应于以上真值表的第三行。
    
设$p$和$q$为两个命题,则$p\leftrightarrow q$表示“$p$等价于$q$”。  

  \begin{tabular}{cc|c}
    p& q& p $\leftrightarrow$ q\\
    \hline
    T&T&T\\
    T&F&F\\
    F&T&F\\
    F&F&T\\
  \end{tabular}

  请大家思考,设$p$,$q$,$r$为命题,则$(p\lor q)\land r$所代表的命题的含义是什么?$(p\land r)\lor (q \land r)$所代表的命题的含义是什么?这两个命题是等价的吗?
  我们可以通过枚举$p$,$q$,$r$依次取值为$T$和$F$时,$(p\lor q)\land r$和$(p\land r)\lor (q \land r)$同时取值为$T$或$F$,从而验证这两个命题是等价的,如下所示:

    \begin{tabular}{ccc|cc}
    $p$& $q$& $r$& $(p\lor q)\land r$&$(p\land r)\lor (q \land r)$\\
    \hline
    T&T&T&T&T\\
    T&F&T&T&T\\
    F&T&T&T&T\\
      F&F&T&F&F\\
    T&T&F&F&F\\
    T&F&F&F&F\\
    F&T&F&F&F\\
      F&F&F&F&F\\      
  \end{tabular}

  用同样的方法我们可以验证:

  $(p\land q)\lor r$与$(p\lor r)\land (q \lor r)$是等价的。

      \begin{tabular}{ccc|cc}
    $p$& $q$& $r$&$(p\land q)\lor r$ &$(p\lor r)\land (q \lor r)$\\
    \hline
    T&T&T&T&T\\
    T&F&T&T&T\\
    F&T&T&T&T\\
      F&F&T&T&T\\
    T&T&F&T&T\\
    T&F&F&F&F\\
    F&T&F&F&F\\
        F&F&F&F&F\\
  \end{tabular}


  $\lnot (p\land q)$与$\lnot p \lor \lnot q$是等价的。

      \begin{tabular}{ccc|cc}
    $p$& $q$& $r$&$\lnot (p\land q)$ &$\lnot p \lor \lnot q$\\
    \hline
    T&T&T&F&F\\
    T&F&T&T&T\\
    F&T&T&T&T\\
      F&F&T&T&T\\      
  \end{tabular}


  $\lnot (p \lor q)$与$\lnot p \land \lnot q$是等价的。
  
      \begin{tabular}{ccc|cc}
    $p$& $q$& $r$&$\lnot (p \lor q)$&$\lnot p \land \lnot q$\\
    \hline
    T&T&T&F&F\\
    T&F&T&F&F\\
    F&T&T&F&F\\
    F&F&T&T&T\\      
  \end{tabular}


\end{CJK}
\end{document}


%%% Local Variables:
%%% mode: latex
%%% TeX-master: t
%%% End:
