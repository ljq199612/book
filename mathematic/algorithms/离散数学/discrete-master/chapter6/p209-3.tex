\documentclass{article}
\usepackage{tikz}
\usepackage{CJKutf8}
\usepackage{amsmath}
\usepackage{amsthm}
\begin{document}
\begin{CJK}{UTF8}{gbsn}
\newtheorem*{Ex}{习题}
  \begin{Ex}
  若$G$是一个$(p,q)$图,$q > \frac{1}{2}(p-1)(p-2)$,试证$G$是连通图。  
  \end{Ex}
  \begin{proof}[证明]
    用反证法。假设$G$不连通,则至少有两个连通分量。设其中一个连通分量的顶点数为
    $p_1$,边数为$q_1$,所有其他连通分量的顶点数为$p_2$,边数为$q_2$。则
    \begin{equation*}
      \begin{split}
        &\frac{1}{2}(p-1)(p-2)\\
        =&\frac{1}{2}(p_1 + p_2 -1)(p_1 + p_2 -2)\\
        =&\frac{1}{2}(p_1 + p_2 -1)((p_1 - 1) + (p_2 - 1))\\
        =&\frac{1}{2}(p_1(p_1 -1) +  p_1(p_2 - 1) + p_2(p_1 - 1) + p_2(p_2-1) - (p_1 - 1) - (p_2-1))\\
        =&\frac{1}{2}(p_1(p_1 -1) +   p_2(p_2-1) + 2(p_1 - 1)(p_2-1))\\
        =&\frac{p_1(p_1 -1)}{2} +   \frac{p_2(p_2-1)}{2} + (p_1 - 1)(p_2-1))\\
        \geq&\frac{p_1(p_1 -1)}{2} +   \frac{p_2(p_2-1)}{2}\\
        \geq & q
      \end{split}
    \end{equation*}
    矛盾。
  \end{proof}
\end{CJK}
\end{document}
%%% Local Variables:
%%% mode: latex
%%% TeX-master: t
%%% End:
